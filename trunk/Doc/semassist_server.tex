%% Semantic Assistants Documentation
%% 
%% This file is part of the Semantic Assistants architecture.
%%
%% Copyright (C) 2009, 2010, 2011 Semantic Software Lab, http://www.semanticsoftware.info
%%
%% The Semantic Assistants architecture is free software: you can
%% redistribute and/or modify it under the terms of the GNU Affero General
%% Public License as published by the Free Software Foundation, either
%% version 3 of the License, or (at your option) any later version.
%% 
%% This program is distributed in the hope that it will be useful,
%% but WITHOUT ANY WARRANTY; without even the implied warranty of
%% MERCHANTABILITY or FITNESS FOR A PARTICULAR PURPOSE.  See the
%% GNU Affero General Public License for more details.
%% 
%% You should have received a copy of the GNU Affero General Public License
%% along with this program.  If not, see <http://www.gnu.org/licenses/>.
%%

\chapter{The \sa Server}
\label{chap:serv}
Semantic NLP services are executed by a Semantic Assistants
server. You can either use an existing server (e.g., from your
company's or university's intranet, or a public server) or run you own
server locally. To access to an existing server, you need to know it's
hostname and port, which are then configured in the client plug-ins
through a preference window.  If you want to run your own server using
the included example NLP services, follow the instructions below.

\section{Starting the Server}
Type \texttt{ant run} in the \url{SemanticAssistants/Server} directory
to start the server. Please refer to Section~\ref{sec:inst-comp} for
more installation and compilation details. The server will
automatically load all available OWL service descriptions from the
default location \url{Resources/OwlServiceDescriptions} and publish
these to the clients.

\subsection{Configuring the Server User Request Limit}
To configure the server to limit the amount of user requests it will
process concurrently, one needs to modify the \texttt{server.thread.allowed}
found in \url{SemanticAssistants/SemassistProperties.xml};  The default
value is already set but if ommitted the number of threads that will
be allowed will not necessarily reflect the server capabilities.

\subsection{Configuring the Server Fixed Pipelines}
The server comes with the possibility to set the number of concurrent threads
of the same pipeline (see GATE documentation for pipeline definition).  To 
configure these settings, one must modfiy/add the following lines
\begin{enumerate}
\item \texttt{server.pipeline.\#.name}
\item \texttt{server.pipeline.\#.number.pooled}
\item \texttt{server.pipeline.\#.max.concurrent}
\item \texttt{server.pipeline.\#.startup}
\item \texttt{server.pipeline.\#.fullpath}
\end{enumerate}
(**note: the '\#' must be replaced by a valid positive integer value)
These lines refer to the pipeline that will be pooled for concurrent and
speedier access.  
``name'' refers to the name of the pipeline as indicated by the contents of the .xgapp file.
``number.pooled'' refers to the number of threads we have allocated for this pipeline.
``max.concurrent'' refers to how many threads you will allow to run concurrently.  This number is not permanent as the pipeline completes the numbers will return to the ``number.pooled'' value
``startup'' indicated by a "true"/"false" string value which loads or ignores the pipeline at server startup.
``fullpath'' refers to the location of the xgapp file that will be used to load the pipeline.
These properties must be grouped and must be a set when being used in the \url{SemanticAssistants/SemassistProperties.xml};
These properties can be omitted if desired and only the \texttt{server.thread.allowed} will be taken into account
One must be aware that the number of pipelines set in \texttt{server.pipeline.\#.number.pooled}
will be substracted from the \texttt{server.thread.allowed} property.  Should the number previously
set in \texttt{server.thread.allowed} not be sufficient, the server will automatically resize the property
to contain the total number of threads needed.

\subsection{Testing the Server by accessing the WSDL}
To test if the Server is operating open your favourite browser and
paste \url{http://<server host>:<server port>/SemAssist?wsdl} (Note
the \texttt{<server host>} has a default value of the local machine
name and the \texttt{<server port>} is the value of the property
\texttt{server.port.wsdl} found in
\url{SemanticAssistants/SemassistProperties.xml}; by default it is set
to 8879.

On most platforms, either \url{http://localhost:8879/SemAssist?wsdl}
or \url{http://127.0.0.1:8879/SemAssist?wsdl} should show you the WSDL
if the server has been initialized correctly.\footnote{If this does
  not work, try substituting the current IP address for
  localhost/127.0.0.1 and check that the port is not blocked by a
  firewall.}

\subsection{Server Testing using the Command Line Client}
To test if the server is running correctly and can be accessed from
the clients, we recommend you run some tests using the command-line
client described in Section~\ref{sec:sacl:clc}.


\section{Integrating New NLP Services}
\label{sec:nlpservices}
For the server to know how to handle the different NLP services
offered through the architecture, it needs a \emph{description} of
each offered service. These are by default located in the
\url{SemanticAssistants/Resources/OwlServiceDescriptions}
directory. The GATE pipelines corresponding to these service
descriptions are located (by default) in
\url{Resources/GatePipelines}. The language service descriptions are
ontologies, building on the \emph{SemanticAssistants.owl} ontology,
which, in turn, extends the \emph{ConceptUpper.owl} ontology. Both of
these are located in \url{ont-repository} under
\url{SemanticAssistants/Resources}.

The details for developing new NLP service descriptions are covered in
Chapter~\ref{chap:services}.  In order to create a new language
service description, it is often easier to copy an old one and edit
it. Prot\'{e}g\'{e}\footnote{Prot\'{e}g\'{e},
  \url{http://protege.stanford.edu}} is helpful as an ontology
editor. Most important is to define the parameters that can be passed
to this language service, as well as the description of the results
that should be passed back to the client.

In summary, to integrate a new NLP service, two steps are necessary:
\begin{enumerate}
\item Store the GATE pipeline implementing the service under
  \url{Resources/GatePipelines} (using GATE's \emph{Save Application State}
  or \emph{Export to Teamware} menu functions).
\item Develop an OWL service description for this pipeline.  For
  details on the OWL NLP description format, please refer to
  Section~\ref{sec:owl}.
\end{enumerate}

\section{\sa RESTful Interface} In addition to the SOAP interface described above, the \sa server also offers a RESTful interface. This server interface conforms to the REpresentational Transfer Protocol\footnote{REST Web Architecture, \url{http://dl.acm.org/citation.cfm?doid=514183.514185}} constraints and allows clients to inquire and invoke NLP services via light-weight HTTP requests. The current supported actions of our RESTful interface are \texttt{GET} and \texttt{POST} requests and its default representation format is XML. Table~\ref{tab:rest_actions} shows the how to use the \sa RESTful web interface.

\begin{table}[htb]
 \centering\small\sffamily
 \begin{tabular}{p{0.25\textwidth}@{\hspace*{4mm}}p{0.07\textwidth}@{\hspace*{4mm}}p{0.6\textwidth}}
   \toprule
   \textbf{URL} & \textbf{Method} & \textbf{Application} \\
   \midrule
   /services & GET &
   \emph{Retrieves the representation of all available assistants} \\

   & & \\

   /service/\{serviceName\} & POST & \emph{Invokes the specified service} \\

   & & \\
   
   /users & GET & \emph{Retrieves the representation of all users (requires admin access)} \\
   
      & & \\

   /user/\{username\} & POST & \emph{Registers a new user} \\

& & \\

   /user/\{usename\} & GET & \emph{Retrieves the representation of the specified user} \\

   \bottomrule
\end{tabular}
 \caption{The \sa RESTful Web Interface}
 \label{tab:rest_actions}
\end{table}
 
 \subsection{Deployment Options}
 The \sa RESTful web interface is essentially a Java Web Archive (WAR) file that like any other Java web application, needs be deployed in a web container. In the following, we describe how the \sa WAR file can be automatically deployed on an Apache Tomcat\footnote{Apache Tomcat Web Server, \url{http://tomcat.apache.org/}} server as well as inside a console.
 
Before proceeding to deploy the server, we have to build the WAR file first. To do so, browse to \url{SemanticAssistants/Server-REST} folder and type \texttt{ant pack}. This command will automatically resolve all the dependencies of the RESTful project, compiles the source code and builds a WAR file, named  \texttt{SemAssistRestlet.war}, in the \url{dist} folder.
 
\subsubsection{Deploying on Apache Tomcat}
Apache Tomcat is an open source web server and a container for Java web applications. The Ant build script in the \url{Server-REST} folder has a special target that automatically deploys the \url{SemAssistRestlet.war} file on a provided Tomcat installation file using the Tomcat's Manager application. The Manager application allows deploying and undeploying various applications without the need to restart the Tomcat server.

To use this option, you need:

\begin{itemize}
\item A Tomcat installation on your system
\item The Tomcat Manager application\footnote{The Manager application is installed by default on context path \texttt{/manager} in Tomcat 5 and later.}
\item Have Tomcat and Manager user credentials in hand\footnote{You can find these credentials in \texttt{\$TOMCAT\_HOME/conf/tomcat-users.xml} file.}
\end{itemize}

Once you have the Tomcat and Manager credentials, proceed to edit the \texttt{build.xml} file in \url{SemanticAssistants/Server-REST} folder. Find the following lines in the file and replace the dummy values with your credentials:

\begin{lstlisting}[language=XML,numbers=left,xleftmargin=8mm,columns=flexible]
    <!-- tomcat properties - replace with correct values -->
    <property name="tomcat.manager.url"       value="http://localhost:8080/manager"/>
    <property name="tomcat.manager.username"       value="test"/>
    <property name="tomcat.manager.password"       value="test"/>
\end{lstlisting}

Now type \texttt{ant deploy} in your console for the Ant script to automatically deploy the \url{SemAssistRestlet.war} file in your Tomcat. You can test the RESTful interface by pointing your browser to \texttt{http://localhost:8080/SemAssistRestlet/services}. If the application is successfully deployed in Tomcat, you should see an XML document listing available services of the connected \sa server.

Similarly, you can stop or undeploy the RESTful interface using \texttt{ant stop} and \texttt{ant undeploy} commands respectively.
 
\subsubsection{Running as a standalone application}
The \url{SemAssistRestlet.war} file comes with an embedded Jetty\footnote{Jetty Web Server, \url{http://jetty.codehaus.org/jetty/}} web server that allows the application to run without the need for an external web container. Embedding the Jetty server inside our application, creates an executable WAR file that requires nothing but a Java runtime environment to run. In order to start the RESTful server, browse to \url{SemanticAssistants/Server-REST/dist} and type \texttt{java -jar SemAssistRestlet.war}. This command will start the RESTful web server on its default port \texttt{8182} and print logs on your console. To test the server, open your web browser and go to \texttt{http://localhost:8182/services}. If the server has started successfully, you should see an XML document in your browser that presents a list all the available services in the connected \sa server.

You can also run the RESTful server on a custom port. To do this, send your preferred port number as an input argument to the executable WAR file by typing \texttt{java -jar SemAssistRestlet.war 1234}. This command will start the server on \texttt{http://localhost:1234}.

\section{Server Authentication}
One of the major concerns when transmitting data in a client-server architecture is the protection of data from being sniffed or hijacked by eavesdroppers. This can be achieved by encrypting the communication channel with cryptographic algorithms. HyperText Transfer Protocol Secure (HTTPS) is a combination of HTTP and Secure Socket Layer (SSL) protocol that uses industry-strength encryption algorithms to secure a channel between a client and a server. In addition, it allows clients to verify the authenticity of the server that they are sending their data to. The \sa RESTful web server also offers a secure interface that is published on the HTTPS protocol. 

To start the secure interface, browse to the \url{SemanticAssistants/Server-REST/dist} folder and type \texttt{java -jar SemAssistRestlet-Secure.war}. Then point your web browser to \texttt{https://localhost:8183/services}. Your browser should prompt you saying that the certificate of this URL is not verified. This is because the certificate that the RESTful web server is presenting to your browser is not verified by a leading SSL Certificate Authority, since it is used in a development environment. You can safely proceed to let your browser accept the certificate and present you an XML document listing the available services of the connected \sa server. 

Similar to the \sa unsecured RESTful interface, you can customize the HTTPS server port number by passing an argument to the WAR file, i.e., \texttt{java -jar SemAssistRestlet-Secure.war 1234}.

Note that the presented approach takes advantage of the embedded Jetty server inside the \sa RESTful web server. If you choose to deploy the RESTful web server on a different web container, please consult the web server's corresponding user guide on how to configure an SSL server.
