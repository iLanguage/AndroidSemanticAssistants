% Semantic Assistants - http://www.semanticsoftware.info/semantic-assistants
%
% This file is part of the Semantic Assistants architecture.
%
% Copyright (C) 2012, 2013 Semantic Software Lab, http://www.semanticsoftware.info
% The Semantic Assistants architecture is free software: you can
% redistribute and/or modify it under the terms of the GNU Affero General
% Public License as published by the Free Software Foundation, either
% version 3 of the License, or (at your option) any later version.
%   
% This program is distributed in the hope that it will be useful,
% but WITHOUT ANY WARRANTY; without even the implied warranty of
% MERCHANTABILITY or FITNESS FOR A PARTICULAR PURPOSE.  See the
% GNU Affero General Public License for more details.
% 
% You should have received a copy of the GNU Affero General Public License
% along with this program.  If not, see <http://www.gnu.org/licenses/>.

\chapter{Liferay Portal Integration}
\label{chap:liferay}
A data portal is a web-based software application, which provides a central entry point to an enormous amount of heterogeneous data sources. By means of personalized access to content, portal data can be adapted to its users and their assigned roles. A portal consists of several portlets, which are displayed as different containers on the webpage. They are independent from each other, but different communication techniques allow the transmission of information from one portlet to another\footnote{For more information please refer to Java Specification Request (JSR 268), ~\url{http://jcp.org/en/jsr/detail?id=268}}.

Portals are often used as enterprise or large-scale information systems to enhance the communication and collaboration among users, such as Yahoo.com\footnote{Yahoo!, \url{http://www.yahoo.com}}. 

The \sa architecture provides a portlet for the Liferay \footnote{Liferay, \url{http://www.liferay.com}} portal, a popular open source portal that allows to connect to a \sa server and employs various NLP services on the content.

\section{Introduction}
In the following sections, we describe the Liferay portal integration in detail, as well as the core, reusable features of the NLP integration.

\subsection{Use Cases of NLP in Liferay Portal}
In the design and implementation of our NLP integration, we took one major use case into account.

\begin{description}
\item[Text Mining Assistants for Portal Users.] 
\item[Indexing Portal Content.] 
\item[Automatic Summarization.]
\end{description}

\noindent
%For some real-world examples on how the \wikinlp integration can be applied to
%various domains, please visit our
%\href{http://www.semanticsoftware.info/semantic-assistants-wiki-nlp-showcase}{showcase}
%web page.

\section{Installation}
The SA-Liferay integration consists of two components: a server-side component and a portlet. The portlet connects to a back-end \sa server in hold of a repository of various NLP pipelines that can be invoked on a given portal content. For the integration of the \sa portlet into your Liferay portal you have two options: On the one hand you are able to download the war-file and to put into the deploy folder of your Liferay distribution. Liferay detects the new component and deploys it, automatically. On the other hand it is possible to download the source files and to build the code in your development environment first, before deploying it. In this part we are going to describe the first option, for the second possibility, please go to the development notes. Additionally, it is necessary to install a new theme and to apply it to the webpage, containing the SemanticAssistants portlet. That theme contains a JavaScript library (jQuery) which is essential for running the \sa portlet properly. Note that the following instructions are based on a Liferay Portal server installation version 6.1.1 CE GA2. 

\subsection{\sa Portlet Deployment}
In order to deploy the \sa portlet into your portal, you have to copy the \texttt{SemanticAssistantsPortlet.war} file into the deploy folder of your Liferay installation. Liferay polls on this folder every few seconds, so it will find the new portlet and deploy it automatically. As soon as the deployment was successful, the new portlet will be available in the general portlet container. Similar to other portlets in Liferay, the \sa portlet can be added to any portal web page. In order to add the \sa portlet, choose \texttt{Add \textgreater~More} from the top menu and then you can find the portlet under the \sa category and click on the \texttt{add} link or drag the portlet name into the desired web page as shown in Figure~\ref{fig:liferay_add_portlet}.

\begin{figure}
\centering
\includegraphics[scale=0.6]{pictures/liferay_add_portlet.png}
\caption{Adding \sa Portlet to a Page}
\label{fig:liferay_add_portlet}
\end{figure}

\subsection{Theme Deployment}
Several functions in the Semantic Assistants portlet need the jquery-libraries for a successful execution. Thus, you have to download and install the SemanticAssistant theme. Go to our build server and a copy the SemanticAssistantTheme.war file into the deploy folder of your Liferay distribution. Liferay automatically detects and deploys the new component. Then go to your Liferay portal, login in and go to that webpage, which contains the SemanticAssitantsPortlet. Select on top \texttt{Manage \textgreater~Page}. On the right side of the opening dialog you see a link to the Look and Feel section. Click on that link and another side appears, where you are able to switch the theme. Select ``Define a specific look and fee'' and scroll down. At the end of the page you will find all available themes including the new SemanticAssistant theme - please select it. Now the theme is applied to your webpage and you can close all administrative dialogs.

\section{User Interface}
Once the portlet is added to the page, you can inquire about available NLP services by selecting a \sa endpoint from the provided combobox. Selecting a server address will dynamically load the list of available services along with their description and runtime parameters. To select an assistant, click on the service's name in order to expand the service information box as shown in Figure~\ref{fig:liferay_sa_portlet}. Optionally, you can customize the execution of the NLP pipelines by modifying their runtime parameters.

\begin{figure}
\centering
\includegraphics[scale=0.6]{pictures/liferay_sa_portlet.png}
\caption{\sa Portlet User Interface in Liferay}
\label{fig:liferay_sa_portlet}
\end{figure}

\blankline

In order to invoke an assistant, you need to first provide the content that is going to be analyzed. Therefore, other portlets in the page must provide a mechanism to initialize specific global variables\footnote{Also known as public rendering parameters in Liferay terminology} that the \sa portlet is listening to. As an example, the \sa integration for Liferay also includes a sample portlet containing an example text for demo purposes. The \emph{SASampleContent} can be found in the \texttt{PATH} and deployed similar to the \sa portlet. When deployed, the SASampleContent portlet provides two buttons, one intended for sending the portlet text to the \sa portlet for analysis and the other for clearing the results from the sample portlet's original content. Figure~\ref{fig:liferay_both_portlets} shows both the aforementioned portlets in a page.

\begin{figure}
\centering
\includegraphics[scale=0.6]{pictures/liferay_both_portlets.png}
\caption{\sa portlet together with a sample content portlet}
\label{fig:liferay_both_portlets}
\end{figure}

You can send the content portlet text for analysis by clicking on the ``Send for Analysis'' button. Then choose a server endpoint address from the \sa portlet combobox and select a desired service to be executed on the provided content. In Figure~\ref{fig:liferay_results_portlet} you can see the ``Person and Location Extractor'' service results ran on the provided content. The offsets of the NLP results are used to highlight the terms in the portlet's original content.

\begin{figure}
\centering
\includegraphics[scale=0.6]{pictures/liferay_results_portlet.png}
\caption{NLP Results highlighted in a sample content portlet}
\label{fig:liferay_results_portlet}
\end{figure}

\section{development notes}
\subsection{directory structure}
\subsection{deploying portlets from source code}
\subsection{communication between portlets}